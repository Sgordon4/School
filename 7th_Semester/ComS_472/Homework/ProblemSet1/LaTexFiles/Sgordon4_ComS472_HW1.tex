\documentclass[12pt]{article}
\usepackage[12pt]{moresize}
\usepackage[margin=1in]{geometry}

\usepackage{amsmath}
\usepackage{amssymb}

\usepackage{graphicx}
\usepackage{subcaption}

\usepackage{multirow} %Combining rows in tables
\usepackage{diagbox}  %Table box split in twain

\usepackage{algorithm}
\usepackage{algpseudocode}
\usepackage{alltt}

\usepackage{multicol}

\usepackage{amssymb} %\checkmark symbol

%\usepackage{hyperref}
%\usepackage[latin1]{inputenc}
%\usepackage{listings}
%\usepackage{scrextend}
%\usepackage{changepage} %Adjustwidth

 

\title{ComS 472\\Homework 1}
\author{Sean Gordon}
\date{Sep 11, 2020}

\begin{document}
\maketitle

- 3.2 - \\

(a) \textbf{States:} 			Glass box holds key or banana\\
\indent\indent \textbf{Initial State:}		Hold key to box 1, outside box 1\\
\indent\indent \textbf{Goal Test:	}		Holding banana\\
\indent\indent \textbf{Actions:}			Open next box using key, move into next box, get contents of current box\\
\indent\indent \textbf{Transition Model:}	A box is opened, a box's contents are obtained\\
\indent\indent \textbf{Cost Function:}		Each action costs 1 unit\\

(b) \textbf{States:}				A sequence of characters\\
\indent\indent \textbf{Initial State:}		Sequence = ABABAECCEC\\
\indent\indent \textbf{Goal Test:	}		Sequence = E\\
\indent\indent \textbf{Actions:}			AC = E, AB = BC, BB = E, and Ex = x for any x\\
\indent\indent \textbf{Transition Model:}	Transform two adjacent characters using an action\\
\indent\indent \textbf{Cost Function:}		Each transformation costs 1 unit\\

(c) \textbf{States:}				Square is painted, unpainted, or is a bottomless pit\\
\indent\indent \textbf{Initial State:}		All squares unpainted or bottomless, Standing on arbitrary square\\
\indent\indent \textbf{Goal Test:	}		All non-bottomless squares painted\\
\indent\indent \textbf{Actions:}			Paint current square, move to adjacent non-bottomless unpainted square\\
\indent\indent \textbf{Transition Model:}	A square is painted, a square is moved to\\
\indent\indent \textbf{Cost Function:}		Each action costs 1 unit\\

(d) \textbf{States:}				3D grid position contains a shipping container or is empty.\\
\indent\indent \textbf{Initial State:}		Each grid space contains a shipping container, crane at (0, 0)\\
\indent\indent \textbf{Goal Test:}			Each grid space is empty\\
\indent\indent \textbf{Actions:}			Move crane to (x,y) coordinate, pick up container under crane coordinate,\\ 
\indent\indent drop container on dock\\
\indent\indent \textbf{Transition Model:}	Crane is moved to (x,y), crane picks up container, \\ 
\indent\indent crane drops container on dock\\
\indent\indent \textbf{Cost Function:}		Each action costs 1 unit



\noindent \hrulefill \\



- 3.4 - \\

(a) \textbf{States:}				Square is red or blue\\
\indent\indent \textbf{Initial State:}		All squares blue\\
\indent\indent \textbf{Goal Test:	}		Each 3x3 sub square is of a differing color than it's neighbors\\
\indent\indent \textbf{Actions:}			Color square, move to next square\\
\indent\indent \textbf{Transition Model:}	Square is colored, moved to next square\\
\indent\indent \textbf{Cost Function:}		Each action costs 1 unit\\
	
\indent\indent State space = $(\#states)^{\#squares}$ = $(2)^{81}$ = 2,417,851,639,229,258,349,412,352\\
\indent\indent (Or is it infinite? The wording has me confused)\\

(b) \textbf{States:} 				Square is red or blue\\
\indent\indent \textbf{Initial State:}		All squares blue\\
\indent\indent \textbf{Goal Test:}		Each 3x3 sub square is of a differing color than it's neighbors\\
\indent\indent \textbf{Actions:}			Color square and move to next square, move to next square\\
\indent\indent \textbf{Transition Model:}	Square is colored and moved to next square, \\
\indent\indent moved to next square\\
\indent\indent \textbf{Cost Function:}		Each action costs 1 unit\\
	
\indent\indent State space = $(\#states)^{\#squares}$ = $(2)^{81}$ = 2,417,851,639,229,258,349,412,352\\
	
\indent\indent BFS would indeed perform faster as the new constraint would remove cycles.\\
\indent\indent Iterative deepening would also perform faster for the same reason.\\
	
(c) \textbf{States:}				Subsquare is red or blue\\
\indent\indent \textbf{Initial State:}		All subsquares blue\\
\indent\indent \textbf{Goal Test:	}		Each subsquare is of a differing 7color than it's neighbors\\
\indent\indent \textbf{Actions:}			Color entire subsquare and move to next subsquare, \\
\indent\indent move to next subsquare\\
\indent\indent \textbf{Transition Model:}	Subsquare is colored and moved to next subsquare, \\
\indent\indent moved to next subsquare\\
\indent\indent \textbf{Cost Function:}		Each action costs 1 unit\\
	
\indent\indent State space = $(\#states)^{\#squares}$ = $(2)^{9}$ = 512\\

(d) 2 solutions\\

(e) i.  C $->$ B: Break each subsquare into 9 squares in a 3x3\\
\indent\indent ii. B $->$ A: Problems A and B have the same solutions\\





\noindent \hrulefill \pagebreak




- 3.7 -\\

A given queen can attack at most 3 spaces in a given column.\\
\indent Therefore, when placing the next queen, there are at least n-3q valid spaces,\\
\indent\ \  where q is the total number of queens already placed.\\
\indent Counting the total number of states: n(n-3)(n-6)(n-9)...(n-3c)\\
\indent With some algebra, this can be simplified into $(n!)^{1/3}$\\

Exhaustive search becomes infeasible when n $>$ 33\\





\noindent \hrulefill \\




- 3.17 - \\

(a) False. The search goal may be at the far bottom left of the search tree, \\
\indent\indent of which DFS will head towards immediately. A* must first go through \\
\indent\indent the surrounding nodes of the starting point.\\

(b) True. The search heuristic is required to never overestimate the cost. \\
\indent\indent This heuristic is almost  guaranteed to underestimate. It will be ineffective, \\
\indent\indent but it will work.\\

(c) False. A* is used commonly in image recognition.\\

(d) True. BFS will find a goal if it exists, and is only concerned with the goal with the \\
\indent\indent shortest depth. As this is unaffected by step cost, BFS remains complete.\\

(e) False. As Manhattan distance requires only one square moved at a time, \\
\indent\indent and a rook can move multiple squares at a time if given space, \\
\indent\indent Manhattan distance may overestimate the cost.




\noindent \hrulefill \\




- 3.22 - \\

If the tree to be searched is a straight line of nodes (A-B-C-D-E-...), \\
\indent BFS and Iterative Deepening will use around the same amount of memory, \\
\indent but BFS will not stop on its path to the goal like Iterative Deepening will.\\
  
  
  
\noindent \hrulefill \\\pagebreak


- 3.27 -\\


\begin{tabular}{|c|c|c|c|c|c|}\hline &&\\[-1em]

Current & Adding & g(n) & h(n) & f(n) & Open Nodes  \\ \hline &&\\[-1em]\hline\\[-1em]

Start		& Lugoj 	& 0 	& 244 	& 244 	& (Lugo, 244) 				\\ \hline &&&&&\\[-1em]
Lugoj		& Timisoara 	& 111 & 329 	& 440 	& (Timi, 440) 				\\ \hline &&&&&\\[-1em]
Lugoj		& Mehadia 	& 70	& 241 	& 301 	& (Timi, 440), (Meha, 301) 		\\ \hline &&&&&\\[-1em]
Mehadia	& Dobreta 	& 145	& 242 	& 387 	& (Timi, 440), (Dobr, 387) 			\\ \hline &&&&&\\[-1em]
Dobreta	& Craiova 	& 265	& 160 	& 425 	& (Timi, 440), (Crai, 425) 			\\ \hline &&&&&\\[-1em]
Craiova	& Rimnicu 	& 411	& 193 	& 604 	& (Timi, 440), (Rimn, 604) 			\\ \hline &&&&&\\[-1em]
Craiova	& Pitesti 	& 403	& 100 	& 503 	& (Timi, 440), (Rimn, 604), (Pite, 503)	\\ \hline &&&&&\\[-1em]
Timisoara	& Arad 	& 229	& 366 	& 595 	& (Rimn, 604), (Pite, 503), (Arad, 595)	\\ \hline &&&&&\\[-1em]
Pitesti		& Bucharest 	& 504	& 0 	& 504 	& (Rimn, 604), (Arad, 595), (Buch, 504)\\ \hline &&&&&\\[-1em]
Bucharest	&   		& 	&  	&  	& (Rimn, 604), (Arad, 595)		\\ \hline
\end{tabular}\\\\





\noindent \hrulefill \\



- 3.31 -\\

\noindent \textbf{Complete:} $0 < w < 2$\\
\noindent \textbf{Optimal:} $0 < w < 1$\\
\noindent \textbf{w==0:} Uninformed best-first search. \textbf{w==1:} A* search. \textbf{w==2:} Greedy best-first search\\





\end{document}

















