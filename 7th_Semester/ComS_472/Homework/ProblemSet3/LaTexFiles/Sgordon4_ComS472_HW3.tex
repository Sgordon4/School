\documentclass[12pt]{article}
\usepackage[12pt]{moresize}
\usepackage[margin=1in]{geometry}

\usepackage{amsmath}
\usepackage{amssymb}

\usepackage{graphicx}
\usepackage{subcaption}

\usepackage{multirow} %Combining rows in tables
\usepackage{diagbox}  %Table box split in twain

\usepackage{algorithm}
\usepackage{algpseudocode}
\usepackage{alltt}

\usepackage{multicol}

\usepackage{amssymb} %\checkmark symbol

%\usepackage{hyperref}
%\usepackage[latin1]{inputenc}
%\usepackage{listings}
%\usepackage{scrextend}
%\usepackage{changepage} %Adjustwidth

  

\title{ComS 472\\Homework 3}
\author{Sean Gordon}
\date{Oct 12, 2020}

\begin{document}
\maketitle


\centerline{- 6.1 - }
\ \\
\noindent 1) 3 colors for (SA), 2 for (WA), 1 for (NT, Q, NSW, V), 3 for (T) $\Rightarrow$ \\\indent $3*2*1*1*1*1*3=18$ solutions\\[.4em]
2) 4:(SA), 3:(WA), 2:(NT, Q, NSW, V), 4:(T) $\Rightarrow$ $4*3*2*2*2*2*4=768$ solutions\\[.4em]
3) 2:(SA), 1:(WA), 0:(NT, Q, NSW, V), 2:(T) $\Rightarrow$ $2*1*0*0*0*0*2=0$ solutions\\[.4em]



\noindent \hrulefill \\



\centerline{- 4.7 - }
\ \\
\noindent \\



\noindent \hrulefill \\



\centerline{- 6.8 - }
\ \\
1) A1 $\rightarrow$ R \ \checkmark\\
2) H \ $\rightarrow$ R \ \textbf{X}, conflicts with A1 $\rightarrow$ backtrack \\
3) H \ $\rightarrow$ G \ \checkmark\\
4) A4 $\rightarrow$ R \ \checkmark\\
5) F1 $\rightarrow$ R \ \checkmark\\
6) A2 $\rightarrow$ R \ \checkmark\\
7) F2 $\rightarrow$ R \ \checkmark\\
8) A3 $\rightarrow$ R \ \checkmark\\
9) T \ $\rightarrow$ R \ \textbf{X}, conflicts with F1 $\rightarrow$ backtrack \\
9) T \ $\rightarrow$ G \ \textbf{X}, conflicts with H \ $\rightarrow$ backtrack \\
9) T \ $\rightarrow$ B \ \checkmark\\



\centerline{- 6.11 - }
\ \\
\noindent \\


\noindent \hrulefill \\



\centerline{- 7.4 - }
\ \\
\noindent 1) Correct: Because false will never be true, it doesn't matter what the second half of the equation is.\\
\noindent 2) Incorrect: True is always true, but False will never be true, so by definition it is incorrect.\\
\noindent 3) 



\noindent \hrulefill \\


\end{document}

















