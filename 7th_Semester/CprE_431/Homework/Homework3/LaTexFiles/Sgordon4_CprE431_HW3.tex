\documentclass[12pt]{article}
\usepackage[12pt]{moresize}
\usepackage[margin=1in]{geometry}

\usepackage{amsmath}
\usepackage{amssymb}

\usepackage{graphicx}
\usepackage{subcaption}

\usepackage{multirow} %Combining rows in tables
\usepackage{diagbox}  %Table box split in twain

\usepackage{algorithm}
\usepackage{algpseudocode}
\usepackage{alltt}

\usepackage{multicol}

\usepackage{amssymb} %\checkmark symbol

%\usepackage{hyperref}
%\usepackage[latin1]{inputenc}
%\usepackage{listings}
%\usepackage{scrextend}
%\usepackage{changepage} %Adjustwidth

 

\title{CprE 431\\Homework 3}
\author{Sean Gordon}
\date{Sep 20, 2020}

\begin{document}
\maketitle


 \begin{enumerate}
    \item DAC 
		\begin{itemize}
		\item Access is determined using the user's identity and ACE entries.
		\item A user is given access to a file by being added to its ACL.
		\item A user can extend permissions it has to other users.
		\end{itemize}
	 MAC
		\begin{itemize}
		\item Access is determined using a user's role.
		\item Access is given ONLY by administrators.\\\\
		\end{itemize}
   
    \item Salt is used simply to change the hash. This stops an attacker from breaking hashes by just generating a single list of hashes for common passwords. An attacker will now need to generate that list for each  salt value as well.\\\\
   
    \item\begin{enumerate}
   	\item High security, but have to change permissions for each file to share with a broader audience.\\
   	Best for a system where files are considered important unless stated otherwise, like a government org.
   	\item Security based on groups, less secure but easier to work with.\\
   	Best for a set of groups which kep their work private from each other, but top security is not necessary, like a large business.
   	\item Low security, but very easy to work with.\\
   	Best where every user is trusted and works closely together, like a small business.\\
   	\end{enumerate}
   	\pagebreak
   
    \item\begin{enumerate}
	\item Each running service takes computer resources, and could be available for use to an attacker.
	\item Information is stored in the /etc directory or the installation tree for specific applications.
	\item It is used to restrict a users view by mapping root to another location, essentially trapping them in that directory and its children. It is fairly easy to escape however, with many 		documented mechanisms.\\\\
   	\end{enumerate}
   	
    \item\begin{enumerate}
	\item Allows system administators to more quickly identify any issues.
	\item Can only tell you what has already happened, and only at certain ranges of data.
	\item Remote logging doesn't require logs to be kept on the system they are created. However, it takes bandwith.
	\item If log files are not rotated, they stay forever and quickly build up space.\\\\
   	\end{enumerate}

    \item \begin{enumerate}
	\item User tries multiple times to log in with different passwords.
	\item User tries accessing unusual websites that have been flagged as malicious.
   	\end{enumerate}
\end{enumerate}


\end{document}
















