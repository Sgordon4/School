\documentclass[12pt]{article}
\usepackage{amsmath}
\usepackage{amssymb}
\usepackage{graphicx}
\usepackage{hyperref}
\usepackage{multicol}
\usepackage[latin1]{inputenc}
\usepackage{listings}
\usepackage{scrextend}
\usepackage{changepage} %Adjustwidth
\usepackage{float}

\usepackage{algorithm}
%\usepackage{algorithmic}
\usepackage{algpseudocode}


\title{ComS 311\\Recitation 3, 2:00 Monday\\Homework 6}
\author{Sean Gordon}
%\date{09/29/2019}

\begin{document}
\maketitle


1) Recurrence: \\
combo = null if remainder $<$ 0\\
combo = [ ] if remainder == 0\\
combo = min( from(i = 0 $\to$ n) {combo(set, remaining - set[i], list)} )\\


\begin{algorithm}[H]
\caption{Find non-negative integers w1, ..., wn.}
\begin{algorithmic} [1]

\State 

\end{algorithmic}
\end{algorithm}


\pagebreak


2) Recurrence: \\
func =  true: spaceLeft == 0 \&\& remainingSum == 0\\
func = false:(spaceLeft == 0 \&\& remainingSum != 0) $||$ index == U.length\\
func = func(U, spaceLeft, remainingSum, index+1) $||$\\
\indent \indent \ \ func(U, spaceLeft-1, remainingSum - U[index], index+1);\\


\begin{algorithm}[H]
\caption{Test for subset of U of size k that adds to T.}
\begin{algorithmic} [1]
\State boolean iterFunc(U, T, k)\{
\State
\State n = U.length;
\State 
\State //Make a 2d array to map subsets
\State matrix[ ][ ] = new boolean[n][T+1];\indent //n, 0$\to$T (not 1$\to$T)
\State 
\State //Make hashmap to track the lengths of each subset that adds to a sum
\State //key = current sum, value = array of subset lengths
\State Hashmap legths = new HashMap$<$Integer, List$<$Integer$>>$();
\State 
\State //Set column 0 to true, as all sums == 0 use empty set
\For{int i = 0; i $<$= n; i++}
\State matrix[i][0] = true;
\EndFor
\State 
\State //For each number in the set
\For{int i = 0; i $<$ n; i++}
\State int number = U[i];
\State 
\State //From 1$\to$T
\For{int sum = 1; sum $<$= T; sum++}
\State//If this number is too big, grab the val above
\If{number $>$ sum}
\State matrix[i][sum] = matrix[i-1][sum];
\State continue;
\EndIf
\State

\algstore{alg2}
\end{algorithmic}
\end{algorithm}

%Split across 2 pages

\begin{algorithm}                     
\begin{algorithmic} [1]
\algrestore{alg2}

\State 
\State //Decide if \# can be added to a prev subset to fit current sum
\State //Use typical subset-sum lookback
\State result1 = matrix[i - 1][sum - number];
\State
\If{result}
\State //We are adding this to the subset
\State //Ex: if sum = 14, number = 9, and lengths@5 = [1, 3, 4],
\State //lengths@14 will now = [2, 4, 5]
\State lengths@sum = lengths@(sum - number)++;
\EndIf
\State
\State //If this number won't fit, we don't add it to the subsets, but
\State //this sum may still be possible, so the matrix should reflect that
\State result2 = result $||$ matrix[i-1][sum];
\State 
\State matrix[i][j] = result2;
\EndFor
\EndFor
\State
\State
\If{! (matrix[n-1][T])}
\State return false;
\EndIf
\State
\State //If there is a possible subset, check that there is one of length k
\State list = lengths@T
\State
\For{int i = 0; i $<$ list.length; i++}
\State if(list[i]== k) return true;
\EndFor
\State return false;
\State\}
\end{algorithmic}
\end{algorithm}


\pagebreak


3) Recurrence:\\
traverse = score if x==M \&\& y==N\\
traverse = max(traverse(maze, M, N, x,\ \ \ \ y+1, score-2),\\
\indent \indent \indent \indent \ \ \ traverse(maze, M, N, x+1, y,\ \ \ \ \ score-2),\\
\indent \indent \indent \indent \ \ \ traverse(maze, M, N, x+1, y+1, score-3))\\



\begin{algorithm}[H]
\caption{Maximize score in M x N maze}
\begin{algorithmic} [1]
\State int iterTraverse(maze, M, N)\{
\State 
\State //Go from left$\to$right, top$\to$bottom, \indent \indent \indent \ \ \ \ [X][X]
\State //looking at the max of cells to left, top left, and top  [X][O]
\State 
\For{int x = 1; x $<$= M; x++}
\For{int y = 1; y $<$= N; y++}
\State //Find largest score for transitioning to this cell
\State int score = max(maze[x-1][y] -2, maze[x-1][y-1] -3, maze[x][y-1] -2)
\State 
\State //Replace maze slot with score + maybe diamond, 
\State // as we no longer need it
\State maze[x][y] += score
\EndFor
\EndFor
\State 
\State return maze[M][N]
\State \}
\end{algorithmic}
\end{algorithm}

%\begin{lstlisting}
%\end{lslisting}
\end{document}




















