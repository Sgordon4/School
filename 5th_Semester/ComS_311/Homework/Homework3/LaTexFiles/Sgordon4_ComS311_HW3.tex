\documentclass[12pt]{article}
\usepackage{amsmath}
\usepackage{amssymb}
\usepackage{graphicx}
\usepackage{hyperref}
\usepackage{multicol}
\usepackage[latin1]{inputenc}
\usepackage{listings}
\usepackage{scrextend}
\usepackage{changepage} %Adjustwidth


% Used for code blocks ----------------------------------------------------------------
\usepackage{color}
\usepackage{xcolor}
\usepackage{listings}

\usepackage{caption}
\DeclareCaptionFont{white}{\color{white}}
\DeclareCaptionFormat{listing}{\colorbox{gray}{\parbox{\textwidth}{#1#2#3}}}
\captionsetup[lstlisting]{format=listing,labelfont=white,textfont=white}
% -----------------------------------------------------------------------------------------


\title{ComS 311\\Recitation 3, 2:00 Monday\\Homework 3}
\author{Sean Gordon}
%\date{09/29/2019}

\begin{document}
\maketitle


\noindent 1a) $T(n) \le 3T(\frac{n}{2}) + Cn^2,\ \ T(2) \le c$\\

\begin{adjustwidth}{.25in}{0in}

$3[3T(\frac{n}{4})+c(\frac{n}{2})^2] + cn^2$\\
$9T(\frac{n}{4})+3c(\frac{n}{2})^2 + cn^2$\\

\noindent
$9[3T(\frac{n}{8})+c(\frac{n}{4})^2]+
3c(\frac{n}{2})^2 + cn^2$\\
\noindent
$27T(\frac{n}{8})+9c(\frac{n}{4})^2+
3c(\frac{n}{2})^2 + cn^2$\\


\noindent
$27[3T(\frac{n}{16})+c(\frac{n}{8})^2]+
9c(\frac{n}{4})^2+
3c(\frac{n}{2})^2 + cn^2$\\
\noindent
$81T(\frac{n}{16})+27c(\frac{n}{8})^2+
9c(\frac{n}{4})^2+
3c(\frac{n}{2})^2 + cn^2$\\


\noindent
$3^4T(\frac{n}{2^4}) + 
3^3c(\frac{n}{2^3})^2 + 
3^2c(\frac{n}{2^2})^2 + 
3^1c(\frac{n}{2^1})^2 + 
3^0c(\frac{n}{2^0})^2$\\
\noindent
$3^4T(\frac{n}{2^4}) + 
\frac{3^3}{2^{3*2}}cn^2 + 
\frac{3^2}{2^{2*2}}cn^2 + 
\frac{3^1}{2^{1*2}}cn^2 + 
\frac{3^0}{2^{0*2}}cn^2$\\

\noindent Final term is $3^kT(\frac{n}{2^k})$. Assuming that $n/2^k =2$ so that T(2) = c\\
$\frac{n}{2^k}=2\ \ \ \Rightarrow\ \ \ n=2^k*2\ \ \ \Rightarrow\ \ \  n=2^{k+1}\ \ \ \Rightarrow\\
log(n)=k+1\ \ \ \Rightarrow\ k=log(n)-1$\\
\indent $\therefore$ end term == $3^{log(n)-1}*c$\\

\noindent Full: $3^{log(n)-1}*c + cn^2\sum_{k=0}^{log(n)-1} (\frac{3}{2^{2}})^k$\\
However, $\lim_{k\to\infty}  (\frac{3}{2^{2}})^k = \frac{1}{1-3/4}$\\
\noindent $\Rightarrow c*3^{log(n)-1}+cn^2 \frac{1}{1-3/4}$

\end{adjustwidth}


\pagebreak


\noindent 1b) $T(n) \le 2T(\frac{n}{2}) + Cnlog(n),\ \ T(2) \le c$\\

\begin{adjustwidth}{.25in}{0in}


$2[2T(\frac{n}{4})+
c(\frac{n}{2})log(\frac{n}{2})] + 
cnlog(n)$\\
\noindent
$4T(\frac{n}{4})+
2c(\frac{n}{2})log(\frac{n}{2}) + 
cnlog(n)$\\


\noindent
$4[2T(\frac{n}{8})+
c(\frac{n}{4})log(\frac{n}{4})] + 
2c(\frac{n}{2})log(\frac{n}{2}) +
cnlog(n)$\\
\noindent
$8T(\frac{n}{8})+
4c(\frac{n}{4})log(\frac{n}{4}) + 
2c(\frac{n}{2})log(\frac{n}{2}) +
cnlog(n)$\\


\noindent
$8[2T(\frac{n}{16})+
c(\frac{n}{8})log(\frac{n}{8})] + 
4c(\frac{n}{4})log(\frac{n}{4}) + 
2c(\frac{n}{2})log(\frac{n}{2}) +
cnlog(n)$\\
\noindent
$16T(\frac{n}{16})+
8c(\frac{n}{8})log(\frac{n}{8}) + 
4c(\frac{n}{4})log(\frac{n}{4}) + 
2c(\frac{n}{2})log(\frac{n}{2}) +
cnlog(n)$\\


End term: $2^kT(\frac{n}{2^k})$ \\
\indent Assuming $\frac{n}{2^k}$ = 2 so that $T(\frac{n}{2^k})=T(2) = c$\\


$\frac{n}{2^k}=2\ \ \ \Rightarrow\ \ \ 
n=2^k*2=2^{k+1}\ \ \ \Rightarrow\\
\indent log(n)=k+1\ \ \ \Rightarrow\ \ \ 
k=log(n)-1$\\


\noindent Full: $2^{log(n)-1}*c + cn \sum_{k=0}^{log(n)-1}log(\frac{n}{2^k})\Rightarrow $\\
\indent $2^{log(n)-1}*c+ cn \sum_{k=0}^{log(n)-1}log(n)+ cn \sum_{k=0}^{log(n)-1}log(2^k)$\\
\indent $2^{log(n)-1}*c+ cnlog(n)+ cn \sum_{k=0}^{log(n)-1}k$\\

\noindent However, $\sum_{k=0}^{log(n)-1}k = \frac{(log(n))(log(n)+1)}{2} = \frac{2log(n)+log(n)}{2}$\\

\noindent$\therefore$ Full: $c*2^{log(n)-1}+ cn\frac{2log(n)+log(n)}{2} \Rightarrow$\\



\end{adjustwidth}


\pagebreak


2) This shit dumb af\\ \\ 

3) To classify purple points, this algorithm will recursively sort the points by their x-coordinate values, then parse the sorted array.\\




%In a set containing points A and B, A is purple if A=(x,y) and B=(x+n, y+m) where n\&m are positive, non-zero values.\\
%As it is is specified that all numbers must be unique

%\begin{lstlisting}
%\end{lslisting}
\end{document}