\documentclass[12pt]{article}
\usepackage{amsmath}
\usepackage{amssymb}
\usepackage{graphicx}
\usepackage{hyperref}
\usepackage{multicol}
\usepackage[latin1]{inputenc}
\usepackage{listings}
\usepackage{scrextend}
\usepackage{changepage} %Adjustwidth
\usepackage[margin=1in]{geometry}


% Used for code blocks ----------------------------------------------------------------
\usepackage{color}
\usepackage{xcolor}
\usepackage{listings}

\usepackage{caption}
\DeclareCaptionFont{white}{\color{white}}
\DeclareCaptionFormat{listing}{\colorbox{gray}{\parbox{\textwidth}{#1#2#3}}}
\captionsetup[lstlisting]{format=listing,labelfont=white,textfont=white}
% -----------------------------------------------------------------------------------------



\title{ComS 342\\Recitation 2, 10:00 Tuesday\\Homework 2}
\author{Sean Gordon}
%\date{09/09/2019}

\begin{document}
\maketitle


\noindent 1.
\begin{adjustwidth}{.25in}{0in}
(define X 88) (define Y 89) (define Z 90)\\
(+ X 32)\\
\$ \$ 120\\
(+ Y 32)\\
\$ 121\\
(+ Z 32)\\
\$ 122\\

\end{adjustwidth}

\noindent 2.
\begin{lstlisting}
(define fib (
	lambda (n) (
		if (> n 1) 
		(+ (fib (- n 1)) (fib (- n 2))) 
		n
	)
))

(define fib (lambda (n) (if (> n 1) (+ (fib (- n 1)) ...
				 (fib (- n 2))) n)))

\end{lstlisting}

\pagebreak


\noindent 3a.
\begin{lstlisting}
(define count (
	lambda (n lst) (
		if(null? lst) 0
		(
			if (= n (car lst)) 
				(+ 1 (count n (cdr lst)))
				(+ 0 (count n (cdr lst)))
		)
	)
))


(define count (lambda (n lst) (if(null? lst) 0(if (= n (car lst)) ...
	(+ 1 (count n (cdr lst)))(+ 0 (count n (cdr lst)))))))

\end{lstlisting}

\pagebreak

\noindent 3b.
\begin{lstlisting}
(define maxhelp (
	lambda (n lst) (
		if(null? lst)
			n
			(if (> n (car lst)) 
				(maxhelp n (cdr lst))
				(maxhelp (car lst) (cdr lst))
			)
	)
))


(define max (
	lambda (lst) (
		if(null? lst) 0 
			(maxhelp(car lst) lst)
	)
))


1.
(define maxhelp (lambda (n lst) (if(null? lst)n(if (> n (car lst)) ...
		(maxhelp n (cdr lst))(maxhelp (car lst) (cdr lst))))))
2.
(define max (lambda (lst) (if(null? lst) 0 (maxhelp(car lst) lst))))
\end{lstlisting}

\pagebreak

\noindent 3c.
\begin{lstlisting}
(define max2 (
	lambda (a b) (
		if(> a b) a b
	)
))


(define maxrepeat (
	lambda (lst) (
		if(null? lst) 0
			(max2 
				(count (car lst) lst)
				(maxrepeat (cdr lst))
			)
	)
))


1.
(define count (lambda (n lst) (if(null? lst) 0(if (= n (car lst)) ...
		(+ 1 (count n (cdr lst)))(+ 0 (count n (cdr lst)))))))
2.
(define max2 (lambda (a b) (if(> a b) a b)))
3.
(define maxrepeat (lambda (lst) (if(null? lst) 0 ...
(max2 (count (car lst) lst)(maxrepeat (cdr lst))))))
\end{lstlisting}

\pagebreak

\noindent 4a.
Should ``cons'' be used here? It is specified this should be done "using list expression"
\begin{lstlisting}
(define pairs (list (list 1 5) (list 6 4) (list 7 8) (list 15 10)))
\end{lstlisting}


\noindent 4b.
\begin{lstlisting}
(define secondSum (
	lambda (lst) (
		+ (car (cdr (car lst))) (
			+ (car (cdr (car (cdr lst)))) (
				+ (car (cdr (car (cdr (cdr lst)))))
				(car (cdr (car (cdr (cdr (cdr lst))))))
			) 
		)
	)
))

(define secondSum (lambda (lst) (+ (car (cdr (car lst))) ...
(+ (car (cdr (car (cdr lst)))) ...
(+ (car (cdr (car (cdr (cdr lst))))) ...
(car (cdr (car (cdr (cdr (cdr lst))))))) ))))
\end{lstlisting}

\pagebreak

\noindent 5a.
\begin{lstlisting}
(define second (
	lambda (apr) (
		apr #f
	)
))

(define second (lambda (apr) (apr #f)))
\end{lstlisting}

\noindent 5b.
Honestly I'm just really confused. There's like 6 ways to do this and nobody has given a clear answer, only cutely dodged questions and withheld important information.
\begin{lstlisting}
(define pair (
    lambda (fst snd) (
        lambda (op) (
            if (= op 1) (+ fst snd) (
                if (= op 2) (- fst snd) (
                    if (= op 3) (* fst snd) (
                        if (= op 4) (/ fst snd)
                                                0
                    )
                )
            )
        )
    )
))

(define pair (lambda (fst snd) (lambda (op) (if (= op 1) ...
(+ fst snd) (if (= op 2) (- fst snd) ...
(if (= op 3) (* fst snd) (if (= op 4) (/ fst snd)0)))))))
\end{lstlisting}

\noindent 5c.
\begin{lstlisting}
(define add (lambda (p) (p 1)))
    ==>
(add apair)
5
\end{lstlisting}

\pagebreak

\noindent 6a.
\begin{lstlisting}
(define mylist (list (list 1 3) (list 4 2) (list 5 6)))
\end{lstlisting}



\noindent 6b.
\begin{lstlisting}
(define applyonnth (
	lambda (op lst n) (
		
		if(null? lst) -1
			(if (= n 1)
				(op (car (car lst)) (car (cdr (car lst))))
				(applyonnth op (cdr lst) (- n 1))
			)
	)
))


(define applyonnth (lambda (op lst n) (if(null? lst) -1 ...
	(if (= n 1)(op (car (car lst)) (car (cdr (car lst)))) ...
	(applyonnth op (cdr lst) (- n 1))))))


Assuming operators are defined as such:
(define add (lambda (x y)(+ x y)))
(define subtract (lambda (x y)(- x y)))
\end{lstlisting}

\noindent 6c.
\begin{lstlisting}
(define applyonnth (
    lambda (op) (
        lambda(lst) (
            lambda(n) (
		
                if(null? lst) -1
                    (if (= n 1)
                        (op (car (car lst)) (car (cdr (car lst))))
                        (((applyonnth op) (cdr lst)) (- n 1))
                    )
                )
            )
        )
))

(define applyonnth (lambda (op) (lambda(lst) (lambda(n) ...
(if(null? lst) -1(if (= n 1)(op (car (car lst)) ...
(car (cdr (car lst))))(((applyonnth op) (cdr lst)) (- n 1))))))))
\end{lstlisting}




%\begin{lstlisting}
%\end{lslisting}
\end{document}