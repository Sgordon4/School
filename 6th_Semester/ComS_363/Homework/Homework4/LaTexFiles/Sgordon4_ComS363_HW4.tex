\documentclass[12pt]{article}
\usepackage[12pt]{moresize}

\usepackage{amsmath}
\usepackage{amssymb}

\usepackage{graphicx}
\usepackage{subcaption}

\usepackage{algorithm}
\usepackage{algpseudocode}
\usepackage{alltt}

\usepackage{booktabs}
\usepackage{multicol}
\usepackage{multirow}
\usepackage[table]{xcolor}

\usepackage[margin=1in]{geometry}

%\usepackage{hyperref}
%\usepackage[latin1]{inputenc}
%\usepackage{listings}
%\usepackage{scrextend}
%\usepackage{changepage} %Adjustwidth


\newenvironment{PTMono}{\fontfamily{PTMono-TLF}\selectfont}{\par}

% LaTeX counter interface for \rownum
% ---
\makeatletter
\@ifundefined{c@rownum}{%
  \let\c@rownum\rownum
}{}
\@ifundefined{therownum}{%
  \def\therownum{\@arabic\rownum}%
}{}
\makeatother


\title{ComS 363\\Homework 4}
\author{Sean Gordon}
\date{April 24, 2020}

\begin{document}
\maketitle


\noindent \hrulefill \\[-.4em]



\noindent 1)\\
\indent (S1) This schedule \textbf{is} serializable, as the result of this schedule is the same as that of \\
\indent \indent running the origional two schedules in series.\\

\indent (S2) This schedule \textbf{is not} serializable, as the result of this schedule is not the same as \\
\indent \indent running the origional two schedules in series: A=0, B=2 rather than A=1, B=2.\\

\indent (S3) This schedule \textbf{is} serializable, as the result of this schedule is the same as that of \\
\indent \indent running the origional two schedules in series.\\[-.4em]


\noindent \hrulefill \\[-.4em]


\noindent 2)\\
\indent (S1) This schedule is neither serial nor strict. It is not serial as both transactions are \\
\indent \indent interleaved, and it is not strict as after T1 reads A, it does not commit or abort before\\
\indent \indent T2 writes A.\\

\indent (S2) This schedule is neither serial nor strict. It is not serial as both transactions are \\
\indent \indent interleaved, and it is not strict as after T1 reads A, it does not commit or abort before\\
\indent \indent T2 writes A.\\

\indent (S3) This schedule is serial and strict. It is serial because the transactions are not \\
\indent \indent interleaved, and is strict because after one transaction reads or writes, a\\
\indent \indent read orwrite on the same value does not occur on a separate transaction\\
\indent \indent until after a commit or abort.\\

\indent (S4) This schedule is neither serial nor strict. It is not serial as both transactions are \\
\indent \indent interleaved, and it is not strict as after T2 writes A, it does not commit or abort \\
\indent \indent before T1 reads A.\\[-.4em]


\hrulefill\\


\noindent 3)\\
\indent(S1)\\[-1.5em]
\rowcolors{2}{gray!25}{white}
\begin{table}[h!]
\centering
\begin{tabular}{c|c|c|c|c|c}
\bottomrule
\rowcolor{gray!50}
Line & Data & Lock & Owner & Waiting & Explanation\\\toprule
1 & A & S & T1 &   & T1 has sucessfully requested an S lock\\
2 & A & S & T1 &   & No change\\
3 & A & X & T1 &   & T1 has sucessfully upgraded to an X lock\\
4 & A & X & T1 &   & No change\\
5 & A &  &  &   & T1 has completed and dissolved its locks\\
6 & A & S & T2 &   & T2 has sucessfully requested an S lock\\
7 & A & S & T2 &   & No change\\
8 & A &  &  &   & T2 has completed and dissolved its locks\\\bottomrule
\end{tabular}
\end{table}\\

\indent(S2)\\[-1.5em]
\begin{table}[h!]
\centering
\begin{tabular}{c|c|c|c|c|c}
\bottomrule
\rowcolor{gray!50}
Line & Data & Lock & Owner & Waiting & Explanation\\\toprule
1 & A & S & T1 &   & T1 has sucessfully requested an S lock\\
2 & A & S & T1 &   & No change\\
3 & A & S & T1, T2 &   & T2 has sucessfully requested an S lock\\
4 & A & S & T1, T2 &   & No change\\

&&&&& T1 has unsuccesfully requested an X lock,\setcounter{rownum}{1}\\
\multirow{-2}{*}{5} & \multirow{-2}{*}{A} &\multirow{-2}{*}{S}&\multirow{-2}{*}{T1, T2}&\multirow{-2}{*}{T1}& and is placed on the waiting list\\

&&&&& T2 has completed and dissolves its locks,\setcounter{rownum}{2}\\
\multirow{-2}{*}{6} & \multirow{-2}{*}{A} &\multirow{-2}{*}{X}&\multirow{-2}{*}{T1}&& allowing T1 to proceed\\

7 & A & X & T1 &   & No change\\
8 & A &  &  &   & T1 has completed and dissolves its locks\\\bottomrule
\end{tabular}
\end{table}\\



\end{document}















