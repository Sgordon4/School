\documentclass[12pt]{article}
\usepackage[12pt]{moresize}

\usepackage{amsmath}
\usepackage{amssymb}

\usepackage{graphicx}
\usepackage{subcaption}

\usepackage{algorithm}
\usepackage{algpseudocode}
\usepackage{alltt}

\usepackage{multicol}

\usepackage[margin=1in]{geometry}

%\usepackage{hyperref}
%\usepackage[latin1]{inputenc}
%\usepackage{listings}
%\usepackage{scrextend}
%\usepackage{changepage} %Adjustwidth


\newenvironment{PTMono}{\fontfamily{PTMono-TLF}\selectfont}{\par}


\title{ComS 363\\Homework 2}
\author{Sean Gordon}
%\date{09/09/2019}

\begin{document}
\maketitle


\hrulefill \\



\noindent 1)\\
\indent \indent (a) A, C, and D should not be used as key, as each has duplicate values in their\\ 
\indent \indent respective columns.\\
\indent \indent B should be used as key as it is the only column without duplicate values.\\

\indent (b) All unique values in C are accompanied by their own corresponding unique values in \\
\indent \indent D, so the dependency is satisfied.\\
\indent \indent $[3 \rightarrow 4]$, $[8 \rightarrow 5]$\\

\indent (c) All unique values in C are \textbf{not} accompanied by their own corresponding unique values\\
\indent \indent in B, so the dependency is \textbf{not} satisfied.\\
\indent \indent $[8 \rightarrow 3]$, $[8 \rightarrow 7]$\\


\hrulefill\\


\noindent 2)\\
\indent \indent (a) AG $\rightarrow$ B $\Rightarrow$ BBB $\rightarrow$ BBCD $\Rightarrow$ BBCDD $\sim>$ BDCBD $\rightarrow$ BDCE $\rightarrow$ BDF\\

\indent \indent (b) B$^+$ = \{B, CD, CE, F\}\\

\indent \indent (c) \textbf{AG} $\rightarrow$ \textbf{B} $\Rightarrow$ BB $\rightarrow$ \textbf{C}B\textbf{D} $\rightarrow$  C\textbf{E} $\rightarrow$ \textbf{F}\\
\indent \indent Starting from AG, all of ABCDEFG can be accessed. Thus, AG is a key.\\


\hrulefill\\
\pagebreak


\noindent 3) \\
\indent \indent (a) \{A $\rightarrow$ B, A $\rightarrow$ C\}\\

\indent \indent (b) \{ABCD $\rightarrow$ E, ABCD $\rightarrow$ F\}\\

\indent \indent (c) \{A $\rightarrow$ B, A $\rightarrow$ C, C $\rightarrow$ D\}\\

\indent \indent (d) \{A $\rightarrow$ B, A $\rightarrow$ C, A $\rightarrow$ D\}\\

\indent \indent (e) \{A $\rightarrow$ B, ACD $\rightarrow$ E, EF $\rightarrow$ G, EF $\rightarrow$ H\}\\


\hrulefill\\


\noindent 4) \\
\indent \indent (a) Disproof:
\begin{tabular}{ccc}
 X & Y & Z\\
 \hline &&\\[-1em]
X1 & Y1 & Z1\\
X1 & Y2 & Z3\\
\end{tabular}\\

\indent \indent (b) \\
\indent \indent \indent 1. X $\rightarrow$ YZ (given)\\
\indent \indent \indent 2. X $\rightarrow$ Y (decomposition) \checkmark\\

\indent \indent (c) Disproof:
\begin{tabular}{cccc}
 W & X & Y & Z\\
 \hline &&\\[-1em]
W1 & X1 & Y1 & Z1\\
W1 & X2 & Y1 & Z2\\
\end{tabular}\\


\hrulefill\\


\noindent 5)\\
\indent \indent (a) Computing attribute closure:
\begin{center}
A $\rightarrow$ A \ \ \ \ \ \ \ \ 
B $\rightarrow$ ABCD \ \ \ \ \ \ \ \  
C $\rightarrow$ CD \ \ \ \ \ \ \ \ 
D $\rightarrow$ D\\

AB $\rightarrow$ ABCD \ \ \ \ \ \ \ \ 
AC $\rightarrow$ ACD \ \ \ \ \ \ \ \ 
AD $\rightarrow$ AD \\

BC $\rightarrow$ ABCD\ \ \ \ \ \ \ \ 
BD $\rightarrow$ ABCD\ \ \ \ \ \ \ \ 
CD $\rightarrow$ CD\\

ABC $\rightarrow$ ABCD\ \ \ \ \ \ \ \ 
ABD $\rightarrow$ ABCD\ \ \ \ \ \ \ \ 
ACD $\rightarrow$ ACD\ \ \ \ \ \ \ \ 
BCD $\rightarrow$ ABCD\\
\end{center}

\indent \indent As all combinations that result in ABCD rely on B, we can conclude that B is the\\
\indent \indent only non-redundant key.\\\\

\indent \indent (b) BC $\rightarrow$ ABCD\\

\indent \indent (c) A $\rightarrow$ A\\


\hrulefill\\
\pagebreak


\noindent 6)
\begin{multicols}{2}
\indent Calculating attribute closure:\\
\indent A $\rightarrow$ A\\
\indent B $\rightarrow$ B\\
\indent C $\rightarrow$ ACD\\
\indent D $\rightarrow$ AD\\

AB $\rightarrow$ ABCD\\
\indent AC $\rightarrow$ ACD\\
\indent AD $\rightarrow$ AD\\

BC $\rightarrow$ ABCD\\
\indent BD $\rightarrow$ ABD\\
\indent CD $\rightarrow$ ACD\\

ABC $\rightarrow$ ABCD\\
\indent ABD $\rightarrow$ ABCD\\
\indent ACD $\rightarrow$ ACD\\
\indent BCD $\rightarrow$ ABCD\\

\columnbreak

%\vspace*{\fill}

\noindent(a) No, because of C $\rightarrow$ D and D $\rightarrow$ A\\

\noindent(b) Start: (ABCD)\\
1. (CAD)(BC) because of C $\rightarrow$ D\\ violation, making C a superkey.\\
2. (DA)(CD)(BC) because of D $\rightarrow$ A\\ violation, making D superkey.\\

\noindent(c) No, the decomposition does not preserve the AB $\rightarrow$ C dependency.\\

\noindent(d) Start: (ABCD)\\
1. (ABC)(CD) because of C $\rightarrow$ D\\ violation, making C a superkey.\\

\noindent (e) No, the decomposition does not preserve the D $\rightarrow$ A dependency.\\


%\vspace*{\fill}
%\vspace*{\fill}

\end{multicols}




\end{document}
















