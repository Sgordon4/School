\documentclass[12pt]{article}
\usepackage[12pt]{moresize}
\usepackage[margin=1in]{geometry}

\usepackage{amsmath}
\usepackage{amssymb}

\usepackage{graphicx}
\usepackage{subcaption}

\usepackage{multirow} %Combining rows in tables
\usepackage{diagbox}  %Table box split in twain

\usepackage{algorithm}
\usepackage{algpseudocode}
\usepackage{alltt}

\usepackage{multicol}

\usepackage{amssymb} %\checkmark symbol

%\usepackage{hyperref}
%\usepackage[latin1]{inputenc}
%\usepackage{listings}
%\usepackage{scrextend}
%\usepackage{changepage} %Adjustwidth

 

\title{Stat 330\\Homework 10}
\author{Sean Gordon}
\date{April 24, 2020}

\begin{document}
\maketitle


\noindent\hrulefill \\[-.8em]


\noindent 1)\\
\indent (a) Mean = 56.9\\
\indent \indent Median = 50.5\\
\indent \indent Q1 = 44.5, Q3 = 58, IQR = 13.5\\
\indent \indent Variance = 632.49, Stand. Dev. = 25.15\\

\indent (b) The only number outside of the range is 130.\\

\indent (c) Mean = 48.8\\
\indent \indent Median = 50\\
\indent \indent Q1 = 44, Q3 = 55, IQR = 11\\
\indent \indent Variance = 43.01, Stand. Dev. = 6.56\\

\indent (d) An outlier will greatly skew the mean and standard deviation, but will not have much\\
\indent \indent effect on the median or IQR.\\


\noindent \hrulefill 


\noindent 2)\\
\indent (a) The histogram is exponential, with the vast majority of diamonds in the lower price \\
\indent \indent range, and a sloping decrease of the number of diamonds as the price increases.\\

\indent (b) Exponential, as the decrease in diamond price follows an exponential curve.\\

\indent (c) As diamond carat increases, the price increases linearly, and the variability \\
\indent \indent increases as well.\\


\noindent \hrulefill \\
\pagebreak


\noindent 3)  ${\Large \frac{1}{n-1}}(n\mathbb{E}(X^2) - n\mathbb{E}(\bar{X}^2))$ = ${\Large \frac{1}{n-1}}n(\mathbb{E}(X^2) - \mathbb{E}(\bar{X}^2))$ = \\[.4em]
\indent  ${\Large \frac{1}{n-1}}n(Var(x) + \mathbb{E}(X)^2 - Var(x) - \mathbb{E}(\bar{X})^2)$ = ${\Large \frac{1}{n-1}}n(\mathbb{E}(X)^2 - \mathbb{E}(\bar{X})^2)$ = \\[.4em]
\indent  ${\Large \frac{1}{n-1}}n(\mathbb{E}(X)^2 - \mathbb{E}(\bar{X})^2)$ = ? = $\sigma^2$\\


\noindent \hrulefill \\


\noindent 4)\\
\indent (a) E({\Large{$\frac{X_1+X_2+X_3+X_4}{4}$}) = $\frac{E(X_1)+E(X_2)+E(X_3)+E(X_4)}{4}$} = {\Large $\frac{4\mu}{4}$} = $\mu$\\[.4em]
\indent \indent E({\Large{$\frac{X_1+2X_2+X_3}{4}$}) = $\frac{E(X_1)+E(X_2)+E(X_2)+E(X_3)}{4}$} = {\Large $\frac{4\mu}{4}$} = $\mu$\\[.4em]

\indent (b) ?


\noindent \hrulefill \\


\noindent 5) \\
\indent (a) E(Y) for Pois($\lambda$) = $\lambda$. \\
\indent \indent $\mu_1 = E(Y) = \bar{Y} = m_1 \ \ \Rightarrow \ \ \lambda = \bar{y} \ \ \Rightarrow\ \  \lambda_{MoM} = \bar{y}$\\

\indent (b){\Large$\prod_{i=1}^{n}  \frac{e^{-\lambda}\lambda^{y_i}}{y_i!}$} = {\Large$ \frac{e^{-n\lambda}\lambda^{ny_i}}{\prod_{i=1}^{n}y_i!}$} $\Rightarrow$\\[.4em]
\indent \indent $l(\lambda) = (-n\lambda + ny\ log(\lambda) ) - \sum_{i=1}^{n}log(y_i!)$\\[.4em]
\indent \indent $dx * l(\lambda) = -n + \frac{ny}{\lambda} = 0 \Rightarrow \lambda_{MLE} = y $\\

\indent (c) $\bar{x} = {\Large\frac{7 + 6 + 7 + 2 + 4}{5}}$ = 5.2 = MoM\\


\noindent \hrulefill \\


\noindent 6)
\indent (a) $\mu = E(X) = \int xf(x)dx \Rightarrow \int_0^1x\theta x^{\theta-1}dx =$ {\Large $\frac{\theta}{\theta+1}$} = $\bar{x}$\\[.4em]
\indent \indent {\Large $\frac{\theta}{\theta+1}$} = $\bar{x} \Rightarrow \theta = ${\Large $\frac{\bar{x}}{1-\bar{x}}$} $\Rightarrow \theta = ${\Large $\frac{.666}{1-.666}$} = 2\\

\indent (b) No idea


\noindent \hrulefill \\


\noindent 7) 




\end{document}
















